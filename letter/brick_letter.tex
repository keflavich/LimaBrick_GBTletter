% Letter on H2CO Brick
%\documentclass[defaultstyle,11pt]{thesis}
%\documentclass[]{report}
%\documentclass[]{article}
%\usepackage{aastex_hack}
%\usepackage{deluxetable}
\documentclass[preprint]{aastex}


%%%%%%%%%%%%%%%%%%%%%%%%%%%%%%%%%%%%%%%%%%%%%%%%%%%%%%%%%%%%%%%%
%%%%%%%%%%%  see documentation for information about  %%%%%%%%%%
%%%%%%%%%%%  the options (11pt, defaultstyle, etc.)   %%%%%%%%%%
%%%%%%%  http://www.colorado.edu/its/docs/latex/thesis/  %%%%%%%
%%%%%%%%%%%%%%%%%%%%%%%%%%%%%%%%%%%%%%%%%%%%%%%%%%%%%%%%%%%%%%%%
%		\documentclass[typewriterstyle]{thesis}
% 		\documentclass[modernstyle]{thesis}
% 		\documentclass[modernstyle,11pt]{thesis}
%	 	\documentclass[modernstyle,12pt]{thesis}

%%%%%%%%%%%%%%%%%%%%%%%%%%%%%%%%%%%%%%%%%%%%%%%%%%%%%%%%%%%%%%%%
%%%%%%%%%%%    load any packages which are needed    %%%%%%%%%%%
%%%%%%%%%%%%%%%%%%%%%%%%%%%%%%%%%%%%%%%%%%%%%%%%%%%%%%%%%%%%%%%%
\usepackage{latexsym}		% to get LASY symbols
\usepackage{graphicx}		% to insert PostScript figures
%\usepackage{deluxetable}
\usepackage{rotating}		% for sideways tables/figures
\usepackage{natbib}  % Requires natbib.sty, available from http://ads.harvard.edu/pubs/bibtex/astronat/
\usepackage{savesym}
\usepackage{amssymb}
%\savesymbol{singlespace}
\savesymbol{doublespace}
%\usepackage{wrapfig}
%\usepackage{setspace}
\usepackage{xspace}
\usepackage{color}
\usepackage{multicol}
\usepackage{mdframed}
\usepackage{url}
\usepackage{subfigure}
%\usepackage{emulateapj}
\usepackage{lscape}
\usepackage{grffile}
\usepackage{standalone}
\standalonetrue
\usepackage{import}
\usepackage[utf8]{inputenc}
\usepackage{longtable}
\usepackage{booktabs}



%%%%%%%%%%%%%%%%%%%%%%%%%%%%%%%%%%%%%%%%%%%%%%%%%%%%%%%%%%%%%%%%
%%%%%%%%%%%%       all the preamble material:       %%%%%%%%%%%%
%%%%%%%%%%%%%%%%%%%%%%%%%%%%%%%%%%%%%%%%%%%%%%%%%%%%%%%%%%%%%%%%

% \title{Star Formation in the Galaxy}
% 
% \author{Adam G.}{Ginsburg}
% 
% \otherdegrees{B.S., Rice University, 2007\\
% 	      M.S., University of Colorado, Boulder, 2009}
% 
% \degree{Doctor of Philosophy}		%  #1 {long descr.}
% 	{Ph.D., Rocket Science (ok, fine, astrophysics)}		%  #2 {short descr.}
% 
% \dept{Department of}			%  #1 {designation}
% 	{Astrophysical and Planetary Sciences}		%  #2 {name}
% 
% \advisor{Prof.}				%  #1 {title}
% 	{John Bally}			%  #2 {name}
% 
% \reader{Prof.~Jeremy Darling}		%  2nd person to sign thesis
% \readerThree{Prof.~Jason Glenn}		%  3rd person to sign thesis
% \readerFour{Prof.~Michael Shull}	%  4rd person to sign thesis
% \readerFour{Prof.~Neal Evans}	%  4rd person to sign thesis
% 
% \abstract{  \OnePageChapter	% one page only ??
% 
%     I discovered dust in space.  
% 
% 	}
% 
% 
% \dedication[Dedication]{	% NEVER use \OnePageChapter here.
% 	To 1, the second number in binary.
% 	}
% 
% \acknowledgements{	\OnePageChapter	% *MUST* BE ONLY ONE PAGE!
% 	All y'all.
% 	}
% 
% \ToCisShort	% a 1-page Table of Contents ??
% 
% \LoFisShort	% a 1-page List of Figures ??
% %	\emptyLoF	% no List of Figures at all ??
% 
% \LoTisShort	% a 1-page List of Tables ??
% %	\emptyLoT	% no List of Tables at all ??
% 
% 
% %%%%%%%%%%%%%%%%%%%%%%%%%%%%%%%%%%%%%%%%%%%%%%%%%%%%%%%%%%%%%%%%%
% %%%%%%%%%%%%%%%       BEGIN DOCUMENT...         %%%%%%%%%%%%%%%%%
% %%%%%%%%%%%%%%%%%%%%%%%%%%%%%%%%%%%%%%%%%%%%%%%%%%%%%%%%%%%%%%%%%
% 
% %%%%  footnote style; default=\arabic  (numbered 1,2,3...)
% %%%%  others:  \roman, \Roman, \alph, \Alph, \fnsymbol
% %	"\fnsymbol" uses asterisk, dagger, double-dagger, etc.
% %	\renewcommand{\thefootnote}{\fnsymbol{footnote}}
% %	\setcounter{footnote}{0}

\newcommand{\paa}{Pa\ensuremath{\alpha}}
\newcommand{\brg}{Br\ensuremath{\gamma}}
\newcommand{\msun}{\ensuremath{M_{\odot}}\xspace}			%  Msun
\newcommand{\mdot}{\ensuremath{\dot{M}}\xspace}
\newcommand{\lsun}{\ensuremath{L_{\odot}}}			%  Lsun
\newcommand{\lbol}{\ensuremath{L_{\mathrm{bol}}}}	%  Lbol
\newcommand{\ks}{K\ensuremath{_{\mathrm{s}}}}		%  Ks
\newcommand{\hh}{\ensuremath{\textrm{H}_{2}}\xspace}			%  H2
\newcommand{\dens}{\ensuremath{n(\hh) [\percc]}\xspace}
\newcommand{\formaldehyde}{\ensuremath{\textrm{H}_2\textrm{CO}}\xspace}
\newcommand{\formaldehydeIso}{\ensuremath{\textrm{H}_2~^{13}\textrm{CO}}\xspace}
\newcommand{\methanol}{\ensuremath{\textrm{CH}_3\textrm{OH}}\xspace}
\newcommand{\ortho}{\ensuremath{\textrm{o-H}_2\textrm{CO}}\xspace}
\newcommand{\oneone}{\ensuremath{1_{10}-1_{11}}\xspace}
\newcommand{\twotwo}{\ensuremath{2_{11}-2_{12}}\xspace}
\newcommand{\threethree}{\ensuremath{3_{12}-3_{13}}\xspace}
\newcommand{\threeohthree}{\ensuremath{3_{03}-2_{02}}\xspace}
\newcommand{\threetwotwo}{\ensuremath{3_{22}-2_{21}}\xspace}
\newcommand{\threetwoone}{\ensuremath{3_{21}-2_{20}}\xspace}
\newcommand{\JKaKc}{\ensuremath{J_{K_a K_c}}}
\newcommand{\water}{H$_{2}$O}		%  H2O
\newcommand{\feii}{\ion{Fe}{2}}		%  FeII
\newcommand{\uchii}{UC\ion{H}{2}\xspace}
\newcommand{\UCHII}{UC\ion{H}{2}\xspace}
\newcommand{\hii}{H~{\sc ii}\xspace}
\newcommand{\Hii}{H~{\sc ii}\xspace}
\newcommand{\HII}{H~{\sc ii}\xspace}
\newcommand{\kms}{\textrm{km~s}\ensuremath{^{-1}}\xspace}	%  km s-1
\newcommand{\nsample}{456\xspace}
\newcommand{\CFR}{5\xspace} % nMPC / 0.25 / 2 (6 for W51 once, 8 for W51 twice) REFEDIT: With f_observed=0.3, becomes 3/2./0.3 = 5
\newcommand{\permyr}{\ensuremath{\mathrm{Myr}^{-1}}\xspace}
\newcommand{\tsuplim}{0.5\xspace} % upper limit on starless timescale
\newcommand{\ncandidates}{18\xspace}
\newcommand{\mindist}{8.7\xspace}
\newcommand{\rcluster}{2.5\xspace}
\newcommand{\ncomplete}{13\xspace}
\newcommand{\middistcut}{13.0\xspace}
\newcommand{\nMPC}{3\xspace} % only count W51 once.  W51, W49, G010
\newcommand{\obsfrac}{30}
\newcommand{\nMPCtot}{10\xspace} % = nmpc / obsfrac
\newcommand{\nMPCtoterr}{6\xspace} % = sqrt(nmpc) / obsfrac
\newcommand{\plaw}{2.1\xspace}
\newcommand{\plawerr}{0.3\xspace}
\newcommand{\mmin}{\ensuremath{10^4~\msun}\xspace}
%\newcommand{\perkmspc}{\textrm{per~km~s}\ensuremath{^{-1}}\textrm{pc}\ensuremath{^{-1}}\xspace}	%  km s-1 pc-1
\newcommand{\kmspc}{\textrm{km~s}\ensuremath{^{-1}}\textrm{pc}\ensuremath{^{-1}}\xspace}	%  km s-1 pc-1
\newcommand{\sqcm}{cm$^{2}$\xspace}		%  cm^2
\newcommand{\percc}{\ensuremath{\textrm{cm}^{-3}}\xspace}
\newcommand{\perpc}{\ensuremath{\textrm{pc}^{-1}}\xspace}
\newcommand{\persc}{\ensuremath{\textrm{cm}^{-2}}\xspace}
\newcommand{\persr}{\ensuremath{\textrm{sr}^{-1}}\xspace}
\newcommand{\peryr}{\ensuremath{\textrm{yr}^{-1}}\xspace}
\newcommand{\perkmspc}{\textrm{per~km~s}\ensuremath{^{-1}}\textrm{pc}\ensuremath{^{-1}}\xspace}	%  km s-1 pc-1
\newcommand{\perkms}{\textrm{per~km~s}\ensuremath{^{-1}}\xspace}	%  km s-1 
\newcommand{\um}{\ensuremath{\mu \textrm{m}}\xspace}    % micron
\newcommand{\mum}{\um}
\newcommand{\htwo}{\ensuremath{\textrm{H}_2}}    % micron
\newcommand{\Htwo}{\ensuremath{\textrm{H}_2}}    % micron
\newcommand{\HtwoO}{\ensuremath{\textrm{H}_2\textrm{O}}}    % micron
\newcommand{\htwoo}{\ensuremath{\textrm{H}_2\textrm{O}}}    % micron
\newcommand{\ha}{\ensuremath{\textrm{H}\alpha}}
\newcommand{\hb}{\ensuremath{\textrm{H}\beta}}
%\newcommand{\so}{ SO~(5~6)-(4~5) }
\newcommand{\regone}{Sh~2-201}
\newcommand{\regtwo}{AFGL~4029}
\newcommand{\regthree}{LW Cas Nebula}
\newcommand{\regfour}{IC 1848}
\newcommand{\regfive}{W5 NW}
\newcommand{\regsix}{SFO 11}
\newcommand{\so}{SO~\ensuremath{5_6-4_5}\xspace}
\newcommand{\SO}{SO~\ensuremath{1_2-1_1}\xspace}
\newcommand{\ammonia}{NH\ensuremath{_3}\xspace}
\newcommand{\twelveco}{\ensuremath{^{12}\textrm{CO}}\xspace}
\newcommand{\thirteenco}{\ensuremath{^{13}\textrm{CO}}\xspace}
\newcommand{\ceighteeno}{\ensuremath{\textrm{C}^{18}\textrm{O}}\xspace}
\def\ee#1{\ensuremath{\times10^{#1}}}
\newcommand{\degrees}{\ensuremath{^{\circ}}}
% can't have \degree because I'm getting a degree...
\newcommand{\lowirac}{800}
\newcommand{\highirac}{8000}
\newcommand{\lowmips}{600}
\newcommand{\highmips}{5000}
\newcommand{\perbeam}{\ensuremath{\textrm{beam}^{-1}}}
\newcommand{\ds}{\ensuremath{\textrm{d}s}}
\newcommand{\dnu}{\ensuremath{\textrm{d}\nu}}
\newcommand{\dv}{\ensuremath{\textrm{d}v}}
\def\secref#1{Section \ref{#1}}
\def\eqref#1{Equation \ref{#1}}
\def\facility#1{#1}
%\newcommand{\arcmin}{'}

\newcommand{\necluster}{Sh~2-233IR~NE}
\newcommand{\swcluster}{Sh~2-233IR~SW}
\newcommand{\region}{IRAS 05358}

\newcommand{\nwfive}{40}
\newcommand{\nouter}{15}

\newcommand{\vone}{{\rm v}1.0\xspace}
\newcommand{\vtwo}{{\rm v}2.0\xspace}
\newcommand\mjysr{\ensuremath{{\rm MJy~sr}^{-1}}}
\newcommand\jybm{\ensuremath{{\rm Jy~bm}^{-1}}}
\newcommand\nbolocat{8552\xspace}
\newcommand\nbolocatnew{548\xspace}
\newcommand\nbolocatnonew{8004\xspace} % = nbolocat-nbolocatnew
\renewcommand\arcdeg{\mbox{$^\circ$}\xspace} 
\renewcommand\arcmin{\mbox{$^\prime$}\xspace} 
\renewcommand\arcsec{\mbox{$^{\prime\prime}$}\xspace} 

\newcommand{\todo}[1]{\textcolor{red}{#1}}
\newcommand{\okinfinal}[1]{{#1}}
%% only needed if not aastex
%\newcommand{\keywords}[1]{}
%\newcommand{\email}[1]{}
%\newcommand{\affil}[1]{}


%aastex hack
%\newcommand\arcdeg{\mbox{$^\circ$}}%
%\newcommand\arcmin{\mbox{$^\prime$}\xspace}%
%\newcommand\arcsec{\mbox{$^{\prime\prime}$}\xspace}%

%\newcommand\epsscale[1]{\gdef\eps@scaling{#1}}
%
%\newcommand\plotone[1]{%
% \typeout{Plotone included the file #1}
% \centering
% \leavevmode
% \includegraphics[width={\eps@scaling\columnwidth}]{#1}%
%}%
%\newcommand\plottwo[2]{{%
% \typeout{Plottwo included the files #1 #2}
% \centering
% \leavevmode
% \columnwidth=.45\columnwidth
% \includegraphics[width={\eps@scaling\columnwidth}]{#1}%
% \hfil
% \includegraphics[width={\eps@scaling\columnwidth}]{#2}%
%}}%


%\newcommand\farcm{\mbox{$.\mkern-4mu^\prime$}}%
%\let\farcm\farcm
%\newcommand\farcs{\mbox{$.\!\!^{\prime\prime}$}}%
%\let\farcs\farcs
%\newcommand\fp{\mbox{$.\!\!^{\scriptscriptstyle\mathrm p}$}}%
%\newcommand\micron{\mbox{$\mu$m}}%
%\def\farcm{%
% \mbox{.\kern -0.7ex\raisebox{.9ex}{\scriptsize$\prime$}}%
%}%
%\def\farcs{%
% \mbox{%
%  \kern  0.13ex.%
%  \kern -0.95ex\raisebox{.9ex}{\scriptsize$\prime\prime$}%
%  \kern -0.1ex%
% }%
%}%

\def\Figure#1#2#3#4#5{
\begin{figure*}[htp]
\includegraphics[scale=#4,angle=#5]{#1}
\caption{#2}
\label{#3}
\end{figure*}
}

% originally intended to be included in a two-column paper
% this is in includegraphics: ,width=3in
% but, not for thesis
\def\OneColFigure#1#2#3#4#5{
\begin{figure}[htpb]
\epsscale{#4}
\includegraphics[scale=#4,angle=#5]{#1}
\caption{#2}
\label{#3}
\end{figure}
}

\def\SubFigure#1#2#3#4#5{
\begin{figure*}[htp]
\addtocounter{figure}{-1}
\epsscale{#4}
\includegraphics[angle=#5]{#1}
\caption{#2}
\label{#3}
\end{figure*}
}

\def\FigureTwo#1#2#3#4#5{
\begin{figure*}[htp]
\epsscale{#5}
\plottwo{#1}{#2}
\caption{#3}
\label{#4}
\end{figure*}
}

\def\FigureTwoAA#1#2#3#4#5#6{
\begin{figure*}[htp]
\subfigure[]{ \includegraphics[scale=#5,width=#6]{#1} }
\\
\subfigure[]{ \includegraphics[scale=#5,width=#6]{#2} }
\caption{#3}
\label{#4}
\end{figure*}
}


\def\TallFigureTwo#1#2#3#4#5#6{
    \FigureTwo{#1}{#2}{#3}{#4}{#5}
    }

\def\SubFigureTwo#1#2#3#4#5{
\begin{figure*}[htp]
\addtocounter{figure}{-1}
\epsscale{#5}
\plottwo{#1}{#2}
\caption{#3}
\label{#4}
\end{figure*}
}

\def\FigureFour#1#2#3#4#5#6{
\begin{figure*}[htp]
\subfigure[]{ \includegraphics[width=3in,type=png,ext=.png,read=.png]{#1} }
\subfigure[]{ \includegraphics[width=3in,type=png,ext=.png,read=.png]{#2} }
\subfigure[]{ \includegraphics[width=3in,type=png,ext=.png,read=.png]{#3} }
\subfigure[]{ \includegraphics[width=3in,type=png,ext=.png,read=.png]{#4} }
\caption{#5}
\label{#6}
\end{figure*}
}

\def\FigureThreePDF#1#2#3#4#5{
\begin{figure*}[htp]
\subfigure[]{ \includegraphics[width=3in,type=pdf,ext=.pdf,read=.pdf]{#1} }
\subfigure[]{ \includegraphics[width=3in,type=pdf,ext=.pdf,read=.pdf]{#2} }
\subfigure[]{ \includegraphics[width=3in,type=pdf,ext=.pdf,read=.pdf]{#3} }
\caption{#4}
\label{#5}
\end{figure*}
}

\def\Table#1#2#3#4#5#6{
%\renewcommand{\thefootnote}{\alph{footnote}}
\begin{deluxetable}{#1}
\tablewidth{0pt}
\tabletypesize{\footnotesize}
\tablecaption{#2}
\tablehead{#3}
\startdata
\label{#4}
#5
\enddata
\bigskip
#6
\end{deluxetable}
%\renewcommand{\thefootnote}{\arabic{footnote}}
}

%\def\tablenotetext#1#2{
%\footnotetext[#1]{#2}
%}

\def\LongTable#1#2#3#4#5#6#7#8{
% required to get tablenotemark to work: http://www2.astro.psu.edu/users/stark/research/psuthesis/longtable.html
\renewcommand{\thefootnote}{\alph{footnote}}
\begin{longtable}{#1}
\caption[#2]{#2}
\label{#4} \\

 \\
\hline 
#3 \\
\hline
\endfirsthead

\hline
#3 \\
\hline
\endhead

\hline
\multicolumn{#8}{r}{{Continued on next page}} \\
\hline
\endfoot

\hline 
\endlastfoot
#7 \\

#5
\hline
#6 \\

\end{longtable}
\renewcommand{\thefootnote}{\arabic{footnote}}
}

\def\TallFigureTwo#1#2#3#4#5#6{
\begin{figure*}[htp]
\epsscale{#5}
\subfigure[]{ \includegraphics[width=#6]{#1} }
\subfigure[]{ \includegraphics[width=#6]{#2} }
\caption{#3}
\label{#4}
\end{figure*}
}

		% file containing author's macro definitions

\begin{document}
% \input{introduction}
% 
% %\input{ch_iras05358}
% \input{ch_w5}
% \input{ch_h2co}
% \input{ch_h2colarge}
% \input{ch_boundhii}
% 
% %\input ch2.tex			% file with Chapter 2 contents
% 
% %%%%%%%%%%%%%%%%%%%%%%%%%%%%%%%%%%%%%%%%%%%%%%%%%%%%%%%%%%%%%%%%%%%
% %%%%%%%%%%%%%%%%%%%%%%%  Bibliography %%%%%%%%%%%%%%%%%%%%%%%%%%%%%
% %%%%%%%%%%%%%%%%%%%%%%%%%%%%%%%%%%%%%%%%%%%%%%%%%%%%%%%%%%%%%%%%%%%
% 
% \bibliographystyle{plain}	% or "siam", or "alpha", or "abbrv"
% 				% see other styles (.bst files) in
% 				% $TEXHOME/texmf/bibtex/bst
% 
% \nocite{*}		% list all refs in database, cited or not.
% 
% \bibliography{thesis}		% bib database file refs.bib
% 
% %%%%%%%%%%%%%%%%%%%%%%%%%%%%%%%%%%%%%%%%%%%%%%%%%%%%%%%%%%%%%%%%%%%
% %%%%%%%%%%%%%%%%%%%%%%%%  Appendices %%%%%%%%%%%%%%%%%%%%%%%%%%%%%%
% %%%%%%%%%%%%%%%%%%%%%%%%%%%%%%%%%%%%%%%%%%%%%%%%%%%%%%%%%%%%%%%%%%%
% 
% \appendix	% don't forget this line if you have appendices!
% 
% %\input appA.tex			% file with Appendix A contents
% %\input appB.tex			% file with Appendix B contents
% 
% %%%%%%%%%%%%%%%%%%%%%%%%%%%%%%%%%%%%%%%%%%%%%%%%%%%%%%%%%%%%%%%%%%%
% %%%%%%%%%%%%%%%%%%%%%%%%   THE END   %%%%%%%%%%%%%%%%%%%%%%%%%%%%%%
% %%%%%%%%%%%%%%%%%%%%%%%%%%%%%%%%%%%%%%%%%%%%%%%%%%%%%%%%%%%%%%%%%%%
% 
% \end{document}
% 
% 


Notes on the GBT observations of ``The Brick'', G0.253+0.016, at 6 cm
(AGBT12B-221) and 2 cm (AGBT14A-110).


Skip to Section \ref{sec:interp} if you want to avoid the calibration details.

\section{Data}

\subsection{Continuum}
Because the \formaldehyde lines are seen in absorption against a continuum
background, it is crucial to determine how much of the observed continuum at 2
and 6 cm is behind The Brick.  A comparison between the radio continuum maps
and the Spitzer 24 \um maps \citep{Yusef-Zadeh2009a} shows a clear
correlation between the two, particularly for the parallel arched filaments
next to the well-known giant radio arc (Figure \ref{fig:cont}).  At the position of The Brick, however,
there is no emission, with surface brightness levels within a factor of 2 of
the off-plane continuum-free regions (the arched filaments are $\sim20\times$
brighter).  It is therefore quite clear that The Brick is in the foreground of
the majority of the radio continuum emission.

The 2 cm continuum shows excellent correlation with the \citet{Law2008a} maps,
albeit at higher spatial resolution.  Some minor artifacts are evident, but
they do not overlap with The Brick and therefore are of minor concern.

At 6 cm, we opt to use the high-quality maps constructed by \citet{Law2008a} in
place of creating our own continuum maps.

\Figure{../figures/continuum_comparison.pdf}
{Comparison of the continuum at different bands
(a) Spitzer 24 \um \citep{Yusef-Zadeh2009a}
(b) GBT C-band \citep{Law2008a}
(c) GBT X-band \citep{Law2008a}
(d) GBT Ku-band (this work)
In panel (d), the labeled green circles show the regions extracted in Section \ref{sec:Spectra}
}{fig:cont}{0.5}{0}

\subsection{Filling Factor}
\label{sec:fillingfactor}
It is evident from maps at multiple wavelengths and the 2 cm \formaldehyde maps
with resolution $\sim50\arcsec$ that The Brick does not fill the GBT 6 beam at
any position.  However, along its long axis, it is weakly resolved, so we need
to derive a separate filling factor for each spectrum extracted from the data
cube.  The angular extent of The Brick along its short axis is $\sim0.7-0.9
\arcmin$.  We approximate it as a rectangle with short axis 0.8\arcmin and long
axis 2\arcmin, which gives a filling factor of 0.75 for a 2.6\arcmin FWHM beam.
We treat this as an upper limit on the filling factor for further analysis, as
the absorbing foreground material may not quite fill the area seen as an IRDC.

\subsection{Comparison of 2 cm and 6 cm data}
Because the data are at different resolution, we smooth the 2 cm data with a
FWHM=2.46 \arcmin gaussian in the image plane.  We then treat the 2 cm data as
if it were observed with a 33m diameter telescope; this approach requires
additional correction for filling factor but is the only sensible way to
compare the two data sets.

\subsection{Spectra}
\label{sec:Spectra}

Off subtraction in C-band was challenging; at many frequencies the entire
mapped area was filled with \formaldehyde \oneone absorption.  Figure
\ref{fig:offs} shows the entire data set (all integrations in a single
polarization) at different percentile cuts and the interpolated off spectrum
eventually used.  Details of the approach are described in Ginsburg et al (in prep).


\Figure{../figures/off_spectra_Cband.pdf}
{A demonstration of the attempts to create a signal-free `off' spectrum for
calibration.  Note that there is significant signal in the median spectrum
(yellow), especially at 0 \kms but also seen spread throughout the band.  Given
the amount of curvature across the observed bandpass, determining an
appropriate continuum is difficult.  The important part is identifying the
continuum \emph{shape} from the data, not the absolute magnitude (as the line
depths do not significantly affect the total power in-band), so we elected to
use the 99th percentile as our off-spectrum and interpolate across lines with
signal remaining in that spectrum.  By selecting the 99th percentile, we
strongly favor the detection of absorption features; we do not expect any
significant emission features and such features are definitely not widespread
or able to affect the observation-averaged spectrum.}
{fig:offs}{0.5}{0}

In the below spectra, the \ammonia spectrum is plotted in normalized units,
while the \formaldehyde spectra are optical depths, i.e.
$\tau=-\log(S_\nu/\bar{S_\nu})$.  Negative optical depths would in principle
indicate emission, however a deep examination of the data reveals that the
baselines are `lumpy' around 35-57 \kms in the \twotwo line, and as shown in
Figure \ref{fig:offs}, there are major difficulties in determining the true
baseline level for the \oneone line.

The three North pointings all suffer from serious negative optical depth issues centered
around 50 \kms, but The Brick is still evident and reasonably well-matched to \ammonia
at $\sim20 \kms$.  In both of the ``tip" regions, the \oneone/\twotwo ratio exceeds 5, indicating
low densities (precision to be added later).

The three ``central'' pointings have modestly lower ratios, but also have clear
\formaldehydeIso detections with a ratio \formaldehyde/\formaldehydeIso
$\sim10-15$, which is moderately lower than the lowest $^{12}$C/$^{13}$C ratios
inferred in the CMZ, $\sim25$ in Sgr B2
\citep{Langer1990a,Savage1992a}\footnote{\citet{Henkel1982a} and
\citet{Henkel1983a} also report a $^{12}$C/$^{13}$C ratio $\sim25$ in Sgr B2
from \formaldehyde, but their approach requires many correction factors}.  

\Figure{../figures/NE_tip_11_22_nh3_spectra.pdf}
{Spectra of the Northeast pointing.  Black: \formaldehyde \oneone, Red:
\formaldehyde \twotwo, Green: \ammonia, Blue: H$_2$ $^{13}$CO \oneone.
Inset: Black shows the \oneone/\twotwo ratio, blue shows the H$_2$ $^{12}$CO/H$_2$ $^{13}$CO \oneone ratio.
NOTE: The }
{fig:spec1}{0.5}{0}
\Figure{../figures/N_center_11_22_nh3_spectra.pdf}
{Spectra of the North-center pointing.  Black: \formaldehyde \oneone, Red: \formaldehyde \twotwo, Green: \ammonia, Blue: H$_2$ $^{13}$CO \oneone}
{fig:spec2}{0.5}{0}
\Figure{../figures/N_tip_11_22_nh3_spectra.pdf}
{Spectra of the North-tip pointing.  Black: \formaldehyde \oneone, Red: \formaldehyde \twotwo, Green: \ammonia, Blue: H$_2$ $^{13}$CO \oneone}
{fig:spec3}{0.5}{0}
\Figure{../figures/SW_tip_11_22_nh3_spectra.pdf}
{Spectra of the Southwest pointing.  Black: \formaldehyde \oneone, Red: \formaldehyde \twotwo, Green: \ammonia, Blue: H$_2$ $^{13}$CO \oneone}
{fig:spec4}{0.5}{0}
\Figure{../figures/S_Center_11_22_nh3_spectra.pdf}
{Spectra of the South-center pointing.  Black: \formaldehyde \oneone, Red: \formaldehyde \twotwo, Green: \ammonia, Blue: H$_2$ $^{13}$CO \oneone}
{fig:spec5}{0.5}{0}

\section{Physical Parameters}
We adopt a velocity gradient $dV/dR = 8 \kms \perpc$, measured from a 50 \kms
velocity difference across the Brick's 6 pc length.

\section{Hasty Interpretation}
\label{sec:interp}
The $\tau$ \oneone/\twotwo ratio varies from $\sim10-15$ at $n\sim10^2-10^3$ \percc to
$\sim2-3$ at $\sim10^4-10^{4.5}$ \percc.  It then drops to 1 above $10^{4.5} \percc$.
See Figure \ref{fig:ratiovsdens}.

% ~/work/h2co/limabean/plotcodes/ratio_to_dens_plots.py
%\Figure{../figures/tau_ratio_vs_density_thinlimit_sigmavary_Xm8.5.pdf}
\Figure{../figures/tau_ratio_vs_density_thinlimit_sigmavary_Xm8.pdf}
{The optical depth ratio as a function of \hh density.  The ratio is shown for
a range of turbulent lognormal density distributions as indicated in the legend;
$\delta$ indicates a delta-function distribution.
The gas temperature is assumed to be 20 K (note: 50 K may be more appropriate, but the changes
are $\sim10\%$), and the \formaldehyde lines
are assumed to both be optically thin (see Section \ref{sec:opticaldepth}).
The X-axis indicates the volume-averaged mean \hh density. }
{fig:ratiovsdens}{0.4}{0}

\FigureTwo{../figures/tau_ratio_vs_density_thinlimit_sigmavary_Xm8_withBrickBGsubdata.pdf}
          {../figures/tau_vs_density_thinlimit_sigmavary_Xm8_withBrickBGsubdata.pdf}
{Optical Depth Ratio (left) and optical depth in each line (right) shown with the
data overlaid.  The data show the observed optical depth and optical depth ratio
at the peak of the Brick \oneone optical depth; the observed optical depth and
optical depth ratio are plotted as partially transparent horizontal lines.  The
opacity of the plotted lines indicates the number of spectral points that lay
at that value.  The optical depth ratio spectrum is shown in Figure
\ref{fig:ratiobgsub}.  The only set of parameters where a single density is
consistent with \emph{both} the \oneone and the \twotwo optical depth is
$n\approx10^{3.7} = 5000$ \percc.  There is a degeneracy between density and
$\sigma_s$, the lognormal density dispersion parameter: Lower turbulence
(smaller $\sigma_s$) would allow for a larger density at a lower
$X(\formaldehyde)$, while higher turbulence requires a lower density and a
higher $X(\formaldehyde)$.
}
{fig:modelswithdata}{1}


In the S Center and SW Center (Figures \ref{fig:spec4} and \ref{fig:spec5}),
the density is higher at higher velocity, peaking on the 50 \kms region.

\subsection{Filling Factor revisited}
The filling factor has an upper limit of 0.75 (Section \ref{sec:fillingfactor}.
Most likely, there is a difference in the filling factor between the \oneone
and \twotwo lines for different portions of the line; the \oneone line probably
traces some `envelope' gas in addition to the denser gas preferentially traced
by \twotwo.  

In the \twotwo line, the integrated image of The Brick is well-represented by
an elliptical gaussian with minor axis FWHM $\sim78$\arcsec.  This
representation yields a geometric area filling factor of 0.58 at the center for
a 2.6\arcmin (156 \arcsec) beam.

Note that differences between filling factors in the two lines driven by the
density distribution - e.g., a model of the \twotwo line tracing a bunch of
``droplets'' in a \oneone ``cloud'' - are accounted for using a turbulent
density distribution function.

\subsection{Optical Depth}
\label{sec:opticaldepth}
It is reasonably likely that significant portions of the cloud are optically
thick in the \oneone line, although the observed optical depth is $\sim 0.15$,
which corrects to $0.39$ when accounting for filling factor, still well below
1.  Any optically thick regions must be subtstructure within The Brick; the
cloud is not optically thick at all positions even at the peak velocity.  The
velocity wings are more likely to be optically thin throughout.

The \twotwo line is unlikely to be optically thick to any significant degree,
with an observed peak optical depth of $\tau\approx0.23$ at 50 \arcsec
resolution (that resolves the cloud reasonably well).

Since the \oneone line may be thick, but the \twotwo cannot be, the
\oneone/\twotwo ratio we measure is a \emph{lower limit}.  There could be
\oneone-absorbing gas we don't see, but we don't miss any \twotwo.

In the most conservative scenario, we are measuring \emph{upper limits} on the
gas density.

\subsection{Density}
The lowest ratio reliably observed is $r\sim1.9$ at $v_{LSR}\approx 43 \kms$,
which corresponds to a density of $n\sim2\ee{4}$\percc \emph{if} one assumes a
$\delta$-function density distribution, i.e. no turbulence, which is
implausible.  Using a turbulent density distribution that is probably typical
of the Galactic disk, a lognormal with width $\sigma_{s} = 1$
\citep{Ginsburg2013a,Federrath2010a}, the density is closer to $n\sim
6\ee{3}$\percc.  Much greater turbulence pushes the volume-averaged density
further down.

These density measurements are significantly inconsistent with the global
averages presented by \citet{Longmore2012b}.  Even in the physically
unreasonable non-turbulent case, the \emph{peak} \formaldehyde density is
limited to be $\sim4\times$ lower than the global average presented by
\citet{Longmore2012b}, and the majority of the \formaldehyde velocity channels
require a substantially ($2-5\times$) lower density.

This difference cannot easily be reconciled.  Even if The Brick is assumed to
be oblate (a hamburger seen edge-on) rather than prolate as assumed in
\citet{Longmore2012b}, the dust-derived average density is $\approx3\ee{4}$
\percc.

If the dust temperature was significantly underestimated, the resulting lower mass 
would bring the densities into closer agreement, but there is no reason to believe
the dust temperatures are incorrect.

\subsubsection{Please take with salt for further consumption}
One possibility that works technically but is physically implausible is that
there are many high-density, compact droplets of gas that contribute
significantly to the dust absorption but are small enough and compact enough to
be optically thick in \formaldehyde \oneone, therefore hiding significant
amounts of gas.  A model in which these are part of a smooth density
distribution does not fit the data, unless some sort of pathological density
distribution is invoked.

Are there any calibration or data issues that could result in the observed low
density?  If the C-band background was significantly underestimated, the \oneone
optical depths would in reality be lower, which would drive up the density.
The \citet{Law2008a} maps used to determine the 6 cm \oneone continuum level
are reliable, though, and we converted them from flux density to surface brightness
appropriately.  Our own C-band map agrees with the \citet{Law2008a} map, albeit
with a significant negative offset; it is possible that we've underestimated
the amplitude of the \formaldehyde absorption (by no more than 20\%), which
goes in the wrong direction.

In the \twotwo line, no previous maps are available, so we are forced to rely
on our own continuum image.  The image quality, as shown in Figure
\ref{fig:cont}, is excellent, but the zero-point is uncertain.  We added an
offset value of 0.4 K to ensure that there were no negative positions in the
continuum map, which is at most a 15\% correction.  The continuum levels around
The Brick in the apertures we selected are near zero anyway, so the continuum
has a small effect here.

Another possibility is that the absorption observed is dominated by `weird'
lines of sight from compact background sources that dominate the total
continuum: this hypothesis is strictly ruled out by the faintness of the
compact sources detected by \citet{Rodriguez2013a}.

Could \formaldehyde emission be to blame?  While no \formaldehyde emission is
detected in our observations, the large beam may be to blame.  Above
$n\sim10^5$ \percc, \formaldehyde goes into emission even above a 10 K
background (which is approximately the brightest background in The Brick; see
Figure 2 in \citet{Darling2012b} for excitation modeling).  Given that the
absorption signature is dominant, \emph{most} of the gas (by mass) must be at a
density below the emission threshold, but that is at least moderately
consistent with the \citet{Longmore2012b} mean density measurement, which is
just shy of $10^5$ \percc.  Another argument against \formaldehyde emission
being significant is the good agreement between the \formaldehyde and \ammonia
2-2 profiles in Figures \ref{fig:spec1}-\ref{fig:spec5}.

There is good motivation to look for \formaldehyde emission with an
interferometer, as the high average density determined by \citet{Longmore2012b}
suggests that such regions should exist within The Brick, and the ALMA data
similarly show some (albeit few) high-density regions.  Nonetheless, it seems
unlikely that emission sources could address the density discrepancy.

\subsubsection{Two distributions?}
Could the \formaldehyde observations be explained by a core-in-cloud model,
where a cloud of $n\sim10^2$ \percc gas dominates the mass and column density,
while a ``core'' of $n\sim10^5$ \percc gas provides the \twotwo absorption?

This hypothesis carries a prediction: to be high-column, low-volume density,
it must be spatially large and therefore should be extended on the sky. This
means we should be able to subtract it off.  I've attempted this by selecting
the high-signal region (basically one beam area) centered on 20 km/s and The Brick
and creating a ``rind" mask (approximately an annulus) to subtract off (see Figure \ref{fig:bgaperture}).

% bgsub_plotting
\Figure{../figures/avgspectrum_ratio_bgsubtraction_smooth.pdf}
{The \formaldehyde \oneone/\twotwo optical depth ratio centered on The Brick
before (red) and after (black) subtracting a background annulus.  The
\twotwo data was smoothed to \oneone resolution prior to background-subtraction.
The aperture definition is shown in Figure \ref{fig:bgaperture}.}
{fig:ratiobgsub}{0.5}{0}

% integrate_brick
% bgsub_image_figure
\Figure{../figures/bgsub_aperture_mask.pdf}
{Image showing the background aperture defined for use in Figure \ref{fig:ratiobgsub}.  The red contour region
shows the background area over which the average spectrum was taken to be ``background''.  
The \formaldehyde figures show the optical depth integrated over the range $-20 < v_{LSR} < 50$ \kms.}
{fig:bgaperture}{0.5}{0}

Performing this subtraction does bring the \oneone/\twotwo ratio down a little,
but only to $\sim2$, as shown in Figure \ref{fig:ratiobgsub}.  That still
requires densities $n<10^{4.2} \percc$, most likely $n<10^4 \percc$.

Another way to look at it is to ask what's left over if we assume that the \twotwo line shows
only gas with $n\approx10^5\percc$.  We can use the fact that $\tau_{\oneone} = \tau_{\twotwo}$ 
at these high densities, and assume that they are still both in absorption, and simply subtract
$\tau_{\twotwo}$ from $\tau_{\oneone}$.  We do this in Figure \ref{fig:oneoneMtwotwo}.

% bgsub_plotting
\Figure{../figures/avgspectrum_11minus22.pdf}
{Spectrum of $\tau_{\oneone} - \tau_{\twotwo}$.  This remaining gas would have an implausibly
high \oneone/\twotwo ratio $ > 26$ using a $3-\sigma$ upper limit on the \twotwo line, so 
this explanation doesn't really work, but if it did it would imply a high-density core with an extremely low-density
surrounding envelope and no continuous distribution between them.}
{fig:oneoneMtwotwo}{0.5}{0}

Another way to look at the data is in the form of a spectral grid (Figure \ref{fig:specgrid}).


\Figure{../figures/spectralgrid_optdepth.pdf}
{Grid of optical depth spectra covering The Brick, with locations labeled.
Black is \formaldehyde \oneone, blue is \formaldehydeIso \oneone, and red is
\formaldehyde \twotwo.  The resolution is 2.5\arcmin but the sampling is
0.5\arcmin, so this map shows only a little more than one full beam.  The -20
to 50 \kms component of The Brick is low density throughout.  The 70 \kms
cloud, by contrast, appears to be modestly higher density.}
{fig:specgrid}{0.6}{0}


\ifstandalone
\bibliographystyle{apj_w_etal}  % or "siam", or "alpha", or "abbrv"
%\bibliography{thesis}      % bib database file refs.bib
\bibliography{bibdesk}      % bib database file refs.bib
\fi

\end{document}
